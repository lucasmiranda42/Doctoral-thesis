This thesis delved into the state of the art of animal motion tracking using key point estimation, and its further analysis using different annotation approaches. Three main goals were defined, including the implementation of novel deep clustering algorithms for the unsupervised analysis of motion tracking data, their deployment alongside other tools as part of a Python package, and their application to characterize a real-world behavioral model.

Along these lines, chapter~\ref{chap:introduction} introduced the broad topics under discussion, including definitions of behavior, history of its quantification and application, and chronic stress as a case study. Moreover, it explored the broad technical foundations for what the thesis aimed to present. 

Chapter~\ref{chap:sota} explored the technical aspects of behavioral analysis in more detail and presented state-of-the-art tools to analyze motion-tracking data in both supervised and unsupervised ways. 

Subsequently, chapter~\ref{chap:methods} explored the methodological aspects of the newly introduced algorithms, and all analyses that were carried out for the presented results. 

Chapter~\ref{chap:joss} then moved to introduce DeepOF, our novel Python package, as published in the Journal of Open Source Software (JOSS). This paper, although short, serves as an entry point to the DeepOF ecosystem, its documentation, and contribution guidelines. Moreover, this publication included a code peer-review process of vital importance to what this thesis aims to stand for: open science, both in terms of methods and code. We believe this small paper is an important milestone for our vision of what DeepOF and similar tools represent in the field. 

Next, chapter~\ref{chap:natcomm} demonstrated how the developed tools can be applied to a real world animal model, such as Chronic Social Defeat Stress. This set of results, published in Nature Communications, explores both supervised and unsupervised pipelines included, and how they yield overlapping yet complementary insights into the shifts in behavior that chronic stress causes in male laboratory mice. We hope this is but a kick-start example of what can be accomplished with this tool, and expect to gain insight into other animal models in the future, both with experiments carried out by direct colleagues and external users.

Finally, chapter~\ref{chap:discussion} attempts to put the presented developments in context, delving into the potential impact of the provided tools in several related fields, such as ecology, integration with other data modalities, and QTL discovery. Moreover, a perspective on how the field is likely to evolve in the near future is discussed.

All in all, the current thesis is but an example of an evolving field, in which (as in many other areas currently powered by machine learning) novel tools are enabling both automation and discovery in ways that were not thought possible a decade ago. When putting these developments in context and as both technology and biological knowledge progress, from single cells to social behavior \cite{Miranda2023IncreasingBehaviors}, the dream of jointly mapping behavioral responses to stimuli in a holistic manner is closer than ever.
\begin{otherlanguage}{german}

% Definition von Verhalten
Versuche, systematisch zu charakterisieren und zu verstehen, wie lebende Organismen auf komplexe Reize reagieren, sind nicht neu. Bis vor kurzem stützten sich die meisten Ansätze jedoch entweder auf Beobachtungsstudien, die es den Forschern unmöglich machen, spezifische Hypothesen zu testen, oder auf allzu vereinfachte und mühsame Laborsituationen, die zu weit von realen Szenarien entfernt sind. In diesem Sinne geht der Trend in letzter Zeit dahin, halbnatürliche Szenarien kontrolliert nachzustellen und neue Technologien einzusetzen, um Informationen von sich frei bewegenden Tieren zu extrahieren, wie z.B. Bewegungsverfolgung, neuronale Aktivität und Vokalisierung.

% DeepLabcut und Bewegungsverfolgung
Auf dem Gebiet der Bewegungsverfolgung und unter Ausnutzung der Fortschritte im Bereich des maschinellen Lernens, insbesondere bei neuronalen Netzen, die für das Computersehen verwendet werden, haben mehrere frei verfügbare Softwarepakete in letzter Zeit begonnen, Tools bereitzustellen, die mit geringem Aufwand eine genaue Verfolgung mehrerer Körperteile über die Zeit ermöglichen, ohne dass physische Marker erforderlich sind. Dies hat Forschern die Möglichkeit eröffnet, mit geringem Aufwand große Datenmengen zu erhalten, was wiederum Entwicklern geholfen hat, neue Wege zu finden, um diese neuartige Datenquelle zu analysieren und Erkenntnisse daraus zu gewinnen.

% Drei Hauptziele
Die vorliegende Arbeit zielt also darauf ab, drei Hauptbeiträge zu leisten. Erstens die Entwicklung neuartiger Deep Clustering-Algorithmen, die speziell auf diese Art von Zeitreihen zugeschnitten sind und mit denen sich das Verhaltensrepertoire von Tieren ohne menschliche Kennzeichnung erforschen lässt. Zweitens, die Bereitstellung dieser Algorithmen in einem Open-Source-Python-Paket, das sie zusammen mit anderen Tools zur Annotation des Verhaltens von Labornagern enthält. Drittens und letztens, die Verwendung der eingesetzten Algorithmen zur Charakterisierung eines realen Tiermodells.

% Einführung in DeepOF
Darüber hinaus handelt es sich um eine publikationsbasierte Dissertation, in der die erzielten Ergebnisse in Form von zwei zur Veröffentlichung in begutachteten Fachzeitschriften angenommenen Artikeln präsentiert werden. Die ersten beiden genannten Ziele werden in einem im Journal of Open Source Software (JOSS) veröffentlichten Artikel mit dem Titel \josstitle behandelt. Hier stellen wir ein ursprünglich entwickeltes Software-Tool namens DeepOF (Deep Open Field) vor, das neben anderen Tools auch mehrere Deep-Clustering-Algorithmen enthält und für Forscher einsatzbereit ist. 

% Einführung in CSDS
Die zweite Arbeit wurde in Nature Communications veröffentlicht und trägt den Titel \natcommtitle. Hier stellen wir eine Charakterisierung von Chronic Social Defeat Stress (CSDS) vor, einem in der Stress- und Depressionsforschung weit verbreiteten Tiermodell, bei dem die im ersten Artikel vorgestellte neue Software zum Einsatz kommt.

% Schlussbemerkungen
Alles in allem liefert diese Arbeit eine Reihe neuartiger Beiträge sowohl zur Verhaltensforschung im Allgemeinen als auch zur Analyse von Motion-Tracking-Daten im Besonderen. In den nächsten Kapiteln werden diese Beiträge im Detail beschrieben und wie ich glaube, dass sie das Potenzial haben, die zukünftige Forschung positiv zu beeinflussen.

\end{otherlanguage}
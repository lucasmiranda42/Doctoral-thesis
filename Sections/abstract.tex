% Definition of behavior
Attempts to systematically characterize and understand how living organisms react to complex stimuli are not new. Until recently, however, most approaches have either relied on observational studies, which prevent researchers from testing specific hypotheses, or in overly simplified and laborious laboratory settings, that are too far from real-world scenarios. Along these lines, a recent trend has been to recreate semi-naturalistic scenarios in a controlled manner, and use new technologies to extract information from freely moving animals such as motion tracking, neural activity, and vocalization, among others.

% DeepLabcut and motion tracking
In the field of motion tracking, and leveraging advances in machine learning, particularly in neural networks used for computer vision, several openly available software packages have recently started to provide tools that require little effort to accurately track multiple body parts over time, without the need for physical markers. This has opened the way for researchers to obtain large amounts of data with little effort, which has in turn helped developers come up with novel ways to analyze and extract insight from this novel data source.

% Three  main goals
Thus and so, the current thesis aims to provide three main contributions. First, to develop novel deep clustering algorithms specifically tailored to this type of time series, that can be used to explore the behavioral repertoire of animals without the need of human labels. Second, the deployment of these algorithms in an open-source Python package, which includes them alongside other tools to annotate the behavior of laboratory rodents. Third and last, to use the deployed algorithms to characterize a real world animal model.

% Introducing DeepOF
Moreover, this is a publication-based dissertation, which presents the accomplished results as two papers accepted for publication in peer-reviewed journals. The first two mentioned goals are addressed in an article published in the Journal of Open Source Software (JOSS), titled \josstitle. Here, we present an originally developed software tool called DeepOF (Deep Open Field), which includes several deep clustering algorithms alongside other tools, and is ready for researchers to use. 

% Introducing CSDS
The second paper was published in Nature Communications, and is titled \natcommtitle. Here, we present a characterization of Chronic Social Defeat Stress (CSDS), an animal model widely used in stress and depression research, using the novel software presented in the first article.

% Closing remarks
All in all, this thesis provides a set of novel contributions to both the behavioral research field in general, and the analysis of motion tracking data in particular. The next few chapters will describe these contributions in detail, and how I believe they hold the potential to positively impact future research.
